\documentclass[10pt]{book}

\usepackage[T2A]{fontenc}
\usepackage{geometry}
\usepackage[utf8]{inputenc}
\usepackage[russian]{babel}

\geometry{
  marginparwidth=12mm,
  papersize={145mm,205mm}, % 145x205 no bleed
  top    = 15mm,           %
  right  = 17mm,           %
  bottom = 25mm,           %
  left   = 23mm            %
}

\usepackage[hyperindex=false]{hyperref}

\setlength{\parskip}{0.32em}

\newcommand{\AUTHOR}[1]{\emph{#1}:}

\newenvironment{teaser}{\vspace{-3em}\slshape}{\vspace{1em}}

\begin{document}

\AUTHOR{Юра Хрусталев} Вспоминается давнее высказывание. В бытность студентом и старшеклассником. С друзьями сидели на скамейке по вечерам. Называли ее "развитие". Мы часто останавливались в "развитии".

\AUTHOR{Антон Чикин} СРОЧНО В НОМЕР. https://twitter.com/WylsacomRed/status/808286758685310976.

\AUTHOR{Ivan Grishaev} Что можно сказать о клиентах, у которых в заголовке Accept-Language указано:Vitaly rules google ☆*:。゜゚・*ヽ(^ᴗ^)ノ*・゜゚。:*☆ ¯\_(ツ)_/¯(ಠ益ಠ)(ಥ‿ಥ)(ʘ‿ʘ)ლ(ಠ_ಠლ)( ͡° ͜ʖ ͡°)ヽ(゚Д゚)ノʕ•̫͡•ʔᶘ ᵒᴥᵒᶅ(=^ ^=)oO?.

\AUTHOR{Ruslan Siraev} Минус один коаоркинг.

\AUTHOR{Vitaly Bolshakov} ?.

\AUTHOR{Ruslan Siraev} Из найденного списка. Ну я три нашел в Воронеже, в один позвонил - не работает.

\AUTHOR{Vitaly Bolshakov} А какие это три?.

\AUTHOR{Ruslan Siraev} Посмотри в истории переписке. В ссылках. Челюскинцев не входил в этот список.

\AUTHOR{Vitaly Bolshakov} Ага, нашел) теперь осталось спросить, какой не работает.

\AUTHOR{Ruslan Siraev} Априори который. Ссылка. Виталий ты тоже в поисках рабочего места себе?).

\AUTHOR{Vitaly Bolshakov} Нет, но интересно, каково состояние среды. Я год назад на челюскинцев работал около месяца. Потом в Самаре в коворкинге сидел. В Самаре было лучше, при том, что в целом IT там слабее развито.

\AUTHOR{Ruslan Siraev} Это понятно.

\AUTHOR{Антон Чикин} Поехали взорвем Самару.

\AUTHOR{Юра Хрусталев} Обычно предлагают начать бомбить Воронеж.

\AUTHOR{Антон Чикин} Воронеж устойчив к кризисам и бомбежкам. Никакие мелкие неприятности не способны серьезно изменить его облик. Будет только больше винтажных зданий для открытия кофеен и барбершопов.

\AUTHOR{Artem Trubachev} Надо кстати в барбершоп сходить.

\AUTHOR{Антон Чикин} Просто балконы на центральных улицах начнут падать чуть чаще. И колеи для объезда ям по газонам будут чуть глубже.

\AUTHOR{Ruslan Siraev} У нас и Gett появился вон.

\AUTHOR{Антон Чикин} Да это все так.

\AUTHOR{Юра Хрусталев} А такси тройка все еще не имеет своего приложения.

\AUTHOR{Антон Чикин} Все равно пока рутакси рулит.

\AUTHOR{Ruslan Siraev} Нахуй тройку). Рутакси уже не рулит.

\AUTHOR{Юра Хрусталев} Это эпоха. Это как сказать нахуй центос5.

\AUTHOR{Mikhail Vyukov} В каком городе?.

\AUTHOR{Mikhail Vyukov} Все равно пока рутакси рулит.

\AUTHOR{Антон Чикин} Если ты конечно не против того, что у водителя зрачки расширены чуть больше чем следовало бы.

\AUTHOR{Ruslan Siraev} )).

\AUTHOR{Антон Чикин} Воронеж, тут его представляет такси "Везёт".

\AUTHOR{Юра Хрусталев} Вывозит.

\AUTHOR{Mikhail Vyukov} Гет поездка 200р, рутакси 280 - один и тот же маршрут -  рулит говоришь?.

\AUTHOR{Юра Хрусталев} А душевный разговор?.

\AUTHOR{Антон Чикин} Это пока Гет спонсирует водителей.

\AUTHOR{Юра Хрусталев} А запах елки и бонда красного.

\AUTHOR{Антон Чикин} И карту скидочную уже заведи, да?.

\AUTHOR{Mikhail Vyukov} Через приложении заказывай, да? это 25\%, скидочная карта 20\%.

\AUTHOR{Антон Чикин} Один раз я попросил водителя подождать, он поехал на заправку и оторвал там шланг. Потом он приехал ко мне посоветоваться что ему делать. Ну или через прилу да.

\AUTHOR{Юра Хрусталев} На втк дело было?.

\AUTHOR{Антон Чикин} Кажись да. На левом берегу, в районе ильича. Я посоветовал ему сдаться властям.

\AUTHOR{Ruslan Siraev} У меня как то водитель рутакси рассказывал как в детсве его брат унижал и родители. Да че только не рассказывали).

\AUTHOR{Юра Хрусталев} А люди этот контент в интернете покупают. А тут фри оф чардж.

\AUTHOR{Антон Чикин} Я строю разговор так, чтобы я рассказывал, а он поддакивал. Именно вот потому, что ты выше написал. Вообщем в рутакси многие заслуживают пяти звездочек. А иногда даже шести.

\AUTHOR{Юра Хрусталев} По 100 бальной то системе.

\AUTHOR{Антон Чикин} Еще был прикол когда у чувака бензин закончился на светофоре. Но в целом надо сказать рутакси довольно неплохо доставляет из точки А в точку Б за небольшие деньги.

\AUTHOR{Антон Власов}

\AUTHOR{Юра Хрусталев} Неудачно ширина подобрана. Еле кликнуть смог. Не работать же.

\AUTHOR{Антон Власов} Шо?.

\AUTHOR{Юра Хрусталев}

\AUTHOR{Антон Чикин}  Немного о том, как наши маки скатываются в сраное говно.

\AUTHOR{Антон Власов} А как кстати ее отрубить? :).

\AUTHOR{Антон Чикин} А ты на SO поищи :))))). http://apple.stackexchange.com/questions/258816/how-to-completely-disable-siri-on-sierra.

\AUTHOR{Антон Власов}

\AUTHOR{Sergey Kharchenko 🦁} А объясните в двух словах человеку, который еще не на Sierra, зачем ее отрубать?.

\AUTHOR{Антон Власов} А нафига она нужна.

\AUTHOR{Антон Чикин} Можешь там лайков поставить.

\AUTHOR{Антон Власов} Не совсем понятно.

\AUTHOR{Юра Хрусталев} Пойду минус тебе въебу за саморекламу.

\AUTHOR{Ruslan Siraev} Вообще должен быть рейтинг у убера и тд только из двух пунктов - доехал / не доехал.

\AUTHOR{Vitaly Bolshakov} Песенки поет.

\AUTHOR{Антон Чикин} Процесс болтается, отжирает память. Юра, я модератору в соседнем чате пожалуюсь. Поставлю ему винишка, и он тебя в ридонли отправит на неделю.

\AUTHOR{Sergey Kharchenko 🦁} 10 Мб памяти?.

\AUTHOR{Антон Чикин} И проц жрал. Короче я не помню, но он меня напрягал.

\AUTHOR{Юра Хрусталев} Ты много пиздишь.

\AUTHOR{Sergey Kharchenko 🦁} Пичаль (.

\AUTHOR{Антон Власов} Мне казалось что он отправляет айдл процессы в своп. И они уже никак не влияют.

\AUTHOR{Юра Хрусталев} От того и проц писал.

\AUTHOR{Антон Чикин} Хотя бы тем, что ты его в настройках отрубаешь, а он сука все равно там висит и что-то делает. Или он шалил как-то... В топе я его что-ли заметил несколько раз. Это нервное.

\AUTHOR{Юра Хрусталев} Я про слежку. Он пишет,когда звук есть в микрофоне. Попробуй заклеить его. Как Марк.

\AUTHOR{Антон Чикин} Я просто руками грязными там берусь. Он уже забился давно.

\AUTHOR{Юра Хрусталев} Кстати же сейчас все понимают, что Марк это не Шатлворт?.

\AUTHOR{Антон Чикин} А кто это?.

\AUTHOR{Юра Хрусталев} Окей. Один космонавт.

\AUTHOR{Ruslan Siraev} Антон ты из деревни вещаешь?).

\AUTHOR{Антон Чикин} Теперь у нас другой космонавт.

\AUTHOR{Ruslan Siraev} Последние 10 лет. Как сделать бекап в слеке?.

\AUTHOR{Антон Чикин} Пойду лайк сам себе поставлю за тонкий юмор.

\AUTHOR{Ruslan Siraev} Нашел.

\AUTHOR{Антон Чикин} Юра, -1 за самопиар, +1 за тонкую шутку, так что по нулям. Очки переходят в зрительный зал.

\AUTHOR{Юра Хрусталев} Очков у него было много, но пользовался он только ...

\AUTHOR{Антон Чикин} И этот тоже?. Ну вообще в линуксе чувствуется отсутствие женской руки. Dd.

\AUTHOR{Антон Чикин} https://techcrunch.com/2017/01/09/atlassian-acquires-trello/.  Твое лицо, когда ты впарил канбан-доску за пол миллиарда.

\AUTHOR{Ruslan Siraev} ).

\AUTHOR{Миша Хорпяков} Не доску а user-base.

\AUTHOR{Ruslan Siraev} И вообще не завидуй).

\AUTHOR{Mikhail Vyukov} Ребят, кто-нибудь чашку грязи заказывал? Как она? Вкусная? Найдено@MissyElli0ttв Баре/ресторане ROUTE 66 в Воронеже.

\AUTHOR{Ruslan Siraev} ).

\AUTHOR{Artem Trubachev} https://pbs.twimg.com/media/C1zuT-fWEAArcLk.jpg:large. Ну я не мог не запостить.

\AUTHOR{Миша Хорпяков}  Зацените дивайс.

\AUTHOR{Artem Trubachev} Зачем такой?. Зачем на клавиатуру смотреть?.

\AUTHOR{Mikhail Vyukov} В кс будешь ебашить, Мишань?.

\AUTHOR{Миша Хорпяков} Говорят такие нужно в год петуха иметь. Чтобы успех был.

\AUTHOR{Artem Trubachev} Кто говорит?.

\AUTHOR{Миша Хорпяков} Внутренний голос подсказывает.

\AUTHOR{Artem Trubachev} А где в Воронеже бургер можно вкусный съесть?. Ну кроме макадака. Раньше у рынка делали, но они закрылись на зиму.

\AUTHOR{Dmitry Chernyshov} Мясной Культ закрылся?.

\AUTHOR{Artem Trubachev} На зиму.

\AUTHOR{Dmitry Chernyshov} Эх.

\AUTHOR{Миша Хорпяков} Ох. Задавать такой вопрос в городе с самым большим количеством бургерных на 1000 житилей в России. * по данным 2GIS.

\AUTHOR{Artem Trubachev} Ну а как?. Если бы была бы одна тогда и спрашивать не надо.

\AUTHOR{Юра Хрусталев} На улицу выйди  и ты сразу наткнешься на бургерную.

\AUTHOR{Artem Trubachev} Так мне не абы какую надо.

\AUTHOR{Миша Хорпяков} Лично мне понравился киоск на Кирова рядом с T-SYSTEMS.

\AUTHOR{Антон Чикин} За Поиском вкусные были.

\AUTHOR{Миша Хорпяков} Круглый, серебристый.

\AUTHOR{Антон Чикин} Еще в мясо о фиш, но говорят он сдулсф.

\AUTHOR{Антон Власов} В охаре епты.

\AUTHOR{Artem Trubachev} @druidvavго в охару.

\AUTHOR{Антон Власов} Нот тудей.

\AUTHOR{Artem Trubachev} Фридай мей би?.

\AUTHOR{Антон Власов} Пресайсли!.

\AUTHOR{Artem Trubachev} Соу би ит.

\AUTHOR{Антон Власов}

\AUTHOR{Антон Чикин} https://pp.vk.me/c637121/v637121119/2c22c/gRPonbe9UfY.jpg. Очень за 300 шутка.

\AUTHOR{Sergey Kharchenko 🦁}

\AUTHOR{Ruslan Siraev} Место бургер в центре. Лучшие бургеры там сейчас. Соседня дверь от Место.

\AUTHOR{Artem Trubachev} http://downtown.ru/voronezh/food/8684. ?.

\AUTHOR{Антон Власов} Место? это же помойка.

\AUTHOR{Антон Чикин} Проект, моноконцепт, *груп. Точно говно какое-то. Надо чтобы тетя Катя на кухне готовила, а Арсен продавал.

\AUTHOR{Миша Хорпяков} Место сейчас норм, после раздела акционеров.

\AUTHOR{Vitaly Bolshakov} На бургеры как раз и разделали.

\AUTHOR{Mikhail Vyukov} Ну с курицей бургер без кетчупа или горчицы, когда с собой берёшь так себе. Суховат.

\AUTHOR{Антон Чикин} Мы про Русапп сейчас говорим?.

\AUTHOR{Ruslan Siraev} Нет. Какой еще блин русапп). У тебя вообще денег нет чтоль?). Может фриланс тебе скидывать?).

\AUTHOR{Юра Хрусталев} Русап сейчас со скидкой можно. Недавно был. Шаверма ниче.

\AUTHOR{Антон Чикин} Я просто не понял зачем еще брать бургер с курицей без соуса.

\AUTHOR{Artem Trubachev} Какого бота не хватает этому чату?. Ваши самые безумные идеи.

\AUTHOR{Vitaly Bolshakov} Бургер-бот.

\AUTHOR{Mikhail Vyukov} Его таким дают по дефолту когда с собой просишь - без кетчупа и тд.

\AUTHOR{Mikhail Vyukov} Я просто не понял зачем еще брать бургер с курицей без соуса.

\AUTHOR{Ruslan Siraev} Бот новостей. Мемом.

\AUTHOR{Антон Чикин} Но с курицей, Холмс!. Бургер. С курицей.

\AUTHOR{Антон Власов} Если в место хорошие бургеры, то в бараке хорошее пиво, а в бахоре лучший в воронеже плов. Как раз чуть выше русапа аудитория :).

\AUTHOR{Юра Хрусталев} Топлю за робин-сдобина.

\AUTHOR{Taras Stotsky} Наиболее близкие к мясному культу - в true burgers сейчас.

\AUTHOR{Антон Чикин} Что такое мясной культ?.

\AUTHOR{Юра Хрусталев} Болеть за спартак?.

\AUTHOR{Антон Чикин} Очередной хипстерский термин, который значит что за кусок еды\одежды\whatever надо заплатить x10? ^))).

\AUTHOR{Mikhail Vyukov} Не ем говядину и тд, только курицу и рыбу.

\AUTHOR{Юра Хрусталев} Как за гитхаб.

\AUTHOR{Taras Stotsky} Это бургерная, которая летом работала. Так называется.

\AUTHOR{Антон Чикин} А.

\AUTHOR{Taras Stotsky} http://downtown.ru/voronezh/food/8384.

\AUTHOR{Юра Хрусталев} Они на 95\% безопастнее мяса?.

\AUTHOR{Mikhail Vyukov} Не, просто не люблю.

\AUTHOR{Антон Чикин} Юр сейчас модно как в детстве: картошку и помидоры я съем, а котлету не хочу.

\AUTHOR{Юра Хрусталев} Кстати, а веганы тут есть?.

\AUTHOR{Антон Чикин} Просто все выросли и пиздить уже некому чтобы доел.

\AUTHOR{Юра Хрусталев} Потому что родители восновном просят просто съехать.

\AUTHOR{Антон Чикин} Бунт поколения 90-х.

\AUTHOR{Ruslan Siraev}

\AUTHOR{Антон Власов} Вот кстати в самом бутике постоянно мясо на грани срока годности впарить пытаются. Типа завтра послезавтра просрочится. На бургеры идет видимо то какое уже продать нельзя :).

\AUTHOR{Ruslan Siraev}

\AUTHOR{Artem Trubachev} Пришел проверить место бургеры.

\AUTHOR{Антон Власов} Бооль.

\AUTHOR{Антон Чикин} Столовка. Че такого-то. На Манхеттене зайди в закусочную - там еще хуже будет. То что на картинке - это прямо USA style как он есть.

\AUTHOR{Artem Trubachev} Чет я был слишком голодный чтобы почувствовать вкус.

\AUTHOR{Ruslan Siraev} Вы там звезду мишлен искали?.

\AUTHOR{Artem Trubachev} Структура конечно на троечку. Разваливается все.

\AUTHOR{Юра Хрусталев} Это ты про руби?.

\AUTHOR{Миша Хорпяков} Он и должен. Фарш разваливается.

\AUTHOR{Artem Trubachev} Лол,. Я тут думаю пхп подучить, кстати. С чего начать?.

\AUTHOR{Юра Хрусталев} С офф сайта, там доки есть.

\AUTHOR{Миша Хорпяков} Начни с web-приложения без фреймворка, с классами. Лучше свой фреймворк написать.

\AUTHOR{Ivan Grishaev} Артем, зачем пхп?.

\AUTHOR{Artem Trubachev} Есть экономические причины.

\AUTHOR{Денис Ковалев} https://pp.vk.me/c637330/v637330551/28d57/H8OZToiJQ2c.jpg.

\AUTHOR{Ivan Grishaev} Все так используют.

\AUTHOR{Юра Хрусталев} Русские СМИ довольно легко пишут о русских компаниях, добившихся успеха за рубежом, но эту новость почему-то пропустили:Сайт MMM стал популярнее чем Facebook в странах Африки! Вообще, если вы не видели нового витка МММ - считайте что вы многое потеряли, там теперь все очень круто, красивые схемы доходности, отказ от национальных валют и работа только с биткоинами, улучшение их МЛМ схемы ну и вообще. Да, и называется это теперь Mavrodi Mondial Movement (MMM). Я бы даже вложился из интереса, но денег жалко :)http://pulse.ng/gist/mmm-nigeria-now-more-popular-than-facebook-nigerians-sign-up-id5786119.html.

\AUTHOR{Юра Хрусталев} Вечер в хату как говорится. https://pp.vk.me/c638220/v638220012/18a81/UE8nY2cFgOI.jpg.

\AUTHOR{Ruslan Siraev} ).

\AUTHOR{Mikhail Vyukov} https://vk.com/wall-29534144_5290671.

\AUTHOR{Mikhail Vyukov}

\AUTHOR{Vitaly Bolshakov} Не все оценят доллар по 30).

\AUTHOR{Mikhail Vyukov} Я знал что кто-то скажет об этом))).

\AUTHOR{Ruslan Siraev} https://ru.aliexpress.com/item/1pc-Newest-Funny-Happy-Man-Guy-Wine-Stopper-Novelty-Bar-Tools-Wine-Cork-Bottle-Plug-Perky/32698674701.html?spm=2114.33020208.13.6.GR9udh&scm=1007.12851.33061.0.

\AUTHOR{Стас М} Идите вы. Нормально же было.

\AUTHOR{Юра Хрусталев} Исправим.

\AUTHOR{Антон Чикин} А что это за кот. ?.

\AUTHOR{Artem Trubachev} Вжух.

\AUTHOR{Антон Чикин} Не понимаю.

\AUTHOR{Юра Хрусталев}

\AUTHOR{Антон Чикин} А, вот так понятнее. Вжух и ты забанен в двух соседних чатах, Юра.

\AUTHOR{Юра Хрусталев} Еще один выпад и я пожалуюсь модератору.

\AUTHOR{Антон Власов}

\AUTHOR{Антон Чикин} Мы с ним кореша, я ему пива поставлю и он меня не забанит. Чет вспомнил. Стою в Рете лет 10 назад, передо мной мужик на кассу, лет 40, битард, покупает мышку. И вот когда продавец почти выписал ему бумажку он выдает главный козырь: "Я зарегистрирован на Большом Воронежском Форуме, ник коля_1965, мне положена скидка 5\%".

\AUTHOR{Mikhail Vyukov}

\AUTHOR{Artem Trubachev} Вечер ахуительных историй.

\AUTHOR{Антон Чикин}

\AUTHOR{Ruslan Siraev}

\AUTHOR{Антон Чикин}

\AUTHOR{Юра Хрусталев} Кстати, недавно просил модераторов поудалять мои сообщения на БВФ.

\AUTHOR{Artem Trubachev} БВФ это что?. Я просто не местный.

\AUTHOR{Ruslan Siraev} А откуда ты?.

\AUTHOR{Антон Чикин} Большой Воронежский Форум.

\AUTHOR{Artem Trubachev} Архангельск.

\AUTHOR{Антон Чикин} С какого района?.

\AUTHOR{Ruslan Siraev} ЭТо как большой архангельский форум.

\AUTHOR{Artem Trubachev} Варавино-Фактория.

\AUTHOR{Ruslan Siraev} Ну вот ты сам все понял).

\AUTHOR{Artem Trubachev} В Архангельске нет форума.

\AUTHOR{Ruslan Siraev} Да что с тобой не так мужик.

\AUTHOR{Антон Чикин} Раньше это место называлосьu-antona.ru. И некоторые даже этого Антона видели. Но потом все скатилось в БВФ.

\AUTHOR{Ruslan Siraev} Раньше называлосьu-antona-chikina.ru?).

\AUTHOR{Антон Чикин} Так этсамое, Юра, продолжай.

\AUTHOR{Юра Хрусталев} Мои сообщения удалил.

\AUTHOR{Artem Trubachev} Никогда не понимал как на форуме можно общаться?.

\AUTHOR{Юра Хрусталев} Они датировались 2004-2007 годами. Там даже мемасики были. И обсуждения пхп.

\AUTHOR{Artem Trubachev} Мемасики в 2004?.

\AUTHOR{Антон Чикин} Продать-купить, шухер всякий там обсуждали, типа аварий и взрывов складов с пиротехникой.

\AUTHOR{Юра Хрусталев} Были локальные мемы. Как про "давай завтра". И интересные личности. Еще был форум домолинка. С торентами. Который давал много пищи. И его модератор f5.

\AUTHOR{Антон Чикин} В основном там было так же, как и на улицах в то время.

\AUTHOR{Artem Trubachev} http://take.ms/Exzlx.

\AUTHOR{Юра Хрусталев} Гандон редкостный.

\AUTHOR{Антон Чикин} F5 я по-моему знавал лично.

\AUTHOR{Антон Власов} Вы еще пидонет вспомните.

\AUTHOR{Антон Чикин} Он любил рассказать какие ножи он покупает для EDC.

\AUTHOR{Юра Хрусталев} Мемасики догда были в рамках смишных картинок.

\AUTHOR{Антон Чикин} Что с Фидонетом?.

\AUTHOR{Антон Власов} С ним все хорошо.

\AUTHOR{Антон Чикин} Я у Тандера на узле сидел.

\AUTHOR{Антон Власов} Но было лучше. Я у забола.

\AUTHOR{Artem Trubachev} Началось.

\AUTHOR{Антон Чикин} Это был пиздатый узел - он работал круглые сутки.

\AUTHOR{Юра Хрусталев} DC. С магнит линками. И скринами в тойже папке. Если вы поняли о чем я.

\AUTHOR{Антон Власов} Кхэ кхэ.

\AUTHOR{Антон Чикин} И винраром все упаковано.

\AUTHOR{Artem Trubachev} Уха.

\AUTHOR{Антон Власов} Еданки. Позвонишь в 3 часа на рол и до 7 качаешь.

\AUTHOR{Mihail Bogdanov} Чет много писанины, не буду все читать ).

\AUTHOR{Антон Чикин} А какие эхи были.

\AUTHOR{Artem Trubachev} https://2ch.hk/ma/src/753809/14778561963400.jpg.

\AUTHOR{Антон Чикин} Сейчас такое говно еще пойди найди в интернете. Артем, да.

\AUTHOR{Антон Власов} Отборное говно.

\AUTHOR{Антон Чикин} Это они.

\AUTHOR{Антон Власов} Лучшее.

\AUTHOR{Антон Чикин} Я думаю если ядерный взрыв будет и ядерная зима - мы на одних архивах фидо лет 20 продержимся.

\AUTHOR{Mihail Bogdanov} Хм.. мне кажется после ядерного взрыва нам будет уже пох. Бомбить с Воронежа ж начнут)))).

\AUTHOR{Антон Чикин} Тут бомбоубежища я манал. Поверь.

\AUTHOR{Юра Хрусталев} https://vk.com/club23816650.

\AUTHOR{Антон Власов} Вижу уже некоторых бомбит, да.

\AUTHOR{Юра Хрусталев} Не благодарите и не проклинайте.

\AUTHOR{Антон Власов} Перегнил да.

\AUTHOR{Антон Чикин} О да, Пилигрим.

\AUTHOR{Artem Trubachev} https://pp.vk.me/c625321/v625321196/43a54/Otv4RlwebcU.jpg. Вместо тысячи слов.

\AUTHOR{Юра Хрусталев} https://pp.vk.me/c903/u14797723/127503323/y_fb070750.jpg. Например. Известная впрошлом личность.

\AUTHOR{Антон Чикин} Даже я его знаю. Хотя я и не маргинал как Юра.

\AUTHOR{Artem Trubachev} Кто? тот чел со стаканом?.

\AUTHOR{Юра Хрусталев} Александр Кузнецов. Его еще Ташкент звали.

\AUTHOR{Антон Чикин} Во, теперь точно вспомнил.

\AUTHOR{Mihail Bogdanov} Хех.

\AUTHOR{Ruslan Siraev} Ему щас лет сто поди.

\AUTHOR{Юра Хрусталев} [просмотрел все фотки с той группы]. Нужно сказать, что собрания рефакторинга мало отличаются.

\AUTHOR{Антон Чикин} Ну нет же. Не говори так.

\AUTHOR{Юра Хрусталев} Направление заданное в середине 200х имеет хороший импульс.

\AUTHOR{Антон Чикин} Мы сохранили дух того времени, но привнесли в него шарм хипстерских лофтов.

\AUTHOR{Ruslan Siraev} У нас пиво дороже.

\AUTHOR{Антон Чикин} Впрочем, если хочешь - летом можно будет на лавочке где-нибудь собраться около переполненной мусорки.

\AUTHOR{Ruslan Siraev} Добро пожаловать на лизюкова ). Жду вас тут каждый вечер , будем отгонять хулиганов. С этих мест.

\AUTHOR{Антон Чикин} Докладчиков напоим до состояния невменяемости. Чтобы было типа мы сидели, тихо пили пиво, а к нам подошел пьяный чувак в шлепанцах на носки и начал что-то задвигать.

\AUTHOR{Artem Trubachev} Норм, я бы пришел.

\AUTHOR{Антон Власов} На горку пионерскую. Рефакторка.

\AUTHOR{Антон Чикин}   У кого еще советский язык включен вконтактике?.

\AUTHOR{Mihail Bogdanov} )))).

\AUTHOR{Антон Чикин}

\AUTHOR{Mikhail Vyukov} https://mailchimp.com/2016. https://www.instagram.com/explore/tags/meowchimp/.

\AUTHOR{Стас М} Я за попойку на лавочке!.

\AUTHOR{Антон Чикин} Раз уж начали, то вот мой 88-й. https://pp.vk.me/c836236/v836236260/1eceb/gLBZrL4g2Ak.jpg. Мне кажется ничего не меняется.

\AUTHOR{Artem Trubachev} Меня в 88м и не было. Ток под конец осени завезли.

\AUTHOR{Стас М} Я даже не планировался ещё в 88.

\AUTHOR{Ruslan Siraev} Топ-10 лучших технологических профессий, согласно US News & World Report:Аналитик компьютерных систем, $86 тысяч, 119 тысяч позиций (появится на рынке труда с 2014 по 2024 год), 2,4\% безработных.Разработчик, $98 тысяч, 135 тысяч, 2\%.ИТ-менеджер, $131 тысяча, 54 тысячи: 1,9\%.Веб-разработчик, $65 тысяч, 40 тысяч, 3,6\%Архитектор компьютерных сетей, $100 тысяч, 13 тысяч, 0,6\%.Администратор баз данных, $82 тысячи, 13 тысяч, 1,0\%.Аналитик систем безопасности, $90 тысяч, 15 тысяч, 3,9\%.Специалист службы поддержки, $62 тысячи, 89 тысяч, 3,7\%.Системный администратор, $78 тысяч, 30 тысяч, 3,1\%.Разработчик программного обеспечения, $80 тысяч, -26500 (спрос впоследствии упадет), 3,5\%.

\AUTHOR{Стас М} Как то дублируются пункты. Разработчик, веб разработчик, разработчик по.

\AUTHOR{Ruslan Siraev}

\AUTHOR{Антон Власов} Господи, вот это баянище.

\AUTHOR{Ruslan Siraev} ). Тут сообщество старперов а не стартаперов.

\AUTHOR{Антон Власов}

\AUTHOR{Ruslan Siraev}

\AUTHOR{Vitaly Bolshakov}

\AUTHOR{Антон Власов}

\AUTHOR{Антон Чикин}  Я выиграл!.

\AUTHOR{Mihail Bogdanov}

\AUTHOR{Юра Хрусталев} Впрочем может и так получится сегодня, облить голову как минимум.

\AUTHOR{Антон Чикин} Юра где ты.

\AUTHOR{Миша Хорпяков} Снег чистит.

\AUTHOR{Vitaly Bolshakov} У нас под окнами паркинг-шоу вот.

\AUTHOR{Миша Хорпяков} У нас тоже регулярно.

\AUTHOR{Антон Чикин} Кито там хотел бизнесhttps://www.facebook.com/oleg.volkov.16/videos/vb.545817853/10154070310487854/?type=2&theater.

\AUTHOR{Artem Trubachev} 2,5 года окупаемость. Чет долго.

\AUTHOR{Vitaly Bolshakov} Да ваще. Надо бы за месяц. Чтобы в плюс. Да побольше.

\AUTHOR{Artem Trubachev} + в промороликах врут раза в два.

\AUTHOR{Миша Хорпяков} А мне нравится. А сколько вложения?. Что-то мне кажется там сумма астрономическая будет.

\AUTHOR{Mikhail Vyukov} https://vk.com/wall-29534144_5304245.

\AUTHOR{Юра Хрусталев} \0.

\AUTHOR{Ruslan Siraev} Коллеги, может кто у нас в воронеже помочь?.  За деньги разумеется. Для АСИ.

\AUTHOR{Vitaly Bolshakov} Кто такая Ася?.

\AUTHOR{Ruslan Siraev} Агентство стратегических инициатив.

\AUTHOR{Mikhail Vyukov} Интернет кто дает? они?.

\AUTHOR{Ruslan Siraev} Это все надо организовать.

\AUTHOR{Mikhail Vyukov} Дай контакт, пообщаемся, если не помогу, подскажу что да как.

\AUTHOR{Ruslan Siraev} Завтра дам Миш. Спрошу у человека не против ли он.

\AUTHOR{Mikhail Vyukov} Ага.

\AUTHOR{Marat Khusnetdinov} Как вообще такие вещи реалезуются, мне интересно просто на чем написано, может стандартное решение есть?.

\AUTHOR{Юра Хрусталев} Wowza, erlyvideo, nimb , YouTube.

\AUTHOR{Vyacheslav Kulakov} NGINX, ivideon. С YouTube и Ivideon свой сервер не нужен. И плеер у них уже есть, вроде как. Если смотреть никто не будет, можно и с камеры сразу транслировать.

\AUTHOR{Marat Khusnetdinov} Вообще народ может кто то помочь с постижением видео стримнга? давно интнресуюсь вопросом.

\AUTHOR{Vyacheslav Kulakov} Чего нужно то?.

\AUTHOR{Marat Khusnetdinov} Ликбез, я ноль в вопросе, а интересно познать, форматы, кодеки, как стривмить, кодировать, ну и сама синхронизация. в общем все с 0.

\AUTHOR{Vyacheslav Kulakov} Как-то слишком абстрактно. Вот справочник с ссылками, с которого можно начать знакомство:http://itmultimedia.ru/spravochnik-po-videotranslyaciyam/. На сайте много интересной инфы, но конкретики маловато - для начала пойдёт.

\AUTHOR{Marat Khusnetdinov} Спасибо, думаю стартануть хатит.

\AUTHOR{Vyacheslav Kulakov} Чуваки эти ещё и ПО делают:http://itmultimedia.ru/veb-reshenie-multimediacam-dlya-ip-videonablyudeniya-vozmozhna-ustanovka-na-svoj-sobstvennyj-server/. В статье этой как раз коротко описано, как всё работает.

\AUTHOR{Юра Хрусталев} Учи mpegdash. Обозримое будущее.

\AUTHOR{Marat Khusnetdinov} По всему пройдусь, просто уж очень интересно познать темную силу стриминга.

\AUTHOR{Иьоравепт иавупитоо} Коворкинг на Челюскинцев.

\AUTHOR{Антон Чикин} Особенно звук сливающейся канализации, который как ничто подчеркивает суть доклада.

\AUTHOR{Миша Хорпяков} Какой-то поломанный бот.

\AUTHOR{Антон Чикин} Реплай не отработал гы.

\AUTHOR{Sergey Kharchenko 🦁} Он старее, чем реплаи.

\AUTHOR{Антон Чикин} https://twitter.com/Alfie_LB/status/820592466042769408.

\AUTHOR{Ruslan Siraev}

\AUTHOR{Юра Хрусталев} "плотное ядро".

\AUTHOR{Artem Trubachev} На сиерру уже можно обновляться?.

\AUTHOR{Ruslan Siraev} Да.

\AUTHOR{Антон Власов} Нет.

\AUTHOR{Artem Trubachev} Окподожду.

\AUTHOR{Ruslan Siraev} Да. Антон а что за причины у тебя?).

\AUTHOR{Vyacheslav Kulakov} Кто-то наш логотип в своих корыстных целях использует:https://pp.vk.me/c837524/v837524262/1e9c5/8gP--eSKpRc.jpg.

\AUTHOR{Антон Чикин}

\AUTHOR{Денис Ковалев} Это не наш логотип. У нас нет логотипа.

\AUTHOR{Антон Чикин} Это Денис не как Атос говорит. А как граф Де Ля Фер.

\AUTHOR{Vyacheslav Kulakov} Так это у нас просто красивая картинка... Надо опять поднять обсуждение...

\AUTHOR{Ruslan Siraev} Давай те опрос сделаем нужен ли новый логотип. ). Но до этого нужно решить каким ботом делать опрос).

\AUTHOR{Vyacheslav Kulakov} Можно ли говорить о новом логотипе, если старого нет.

\AUTHOR{Artem Trubachev} Не нужен.

\AUTHOR{Vyacheslav Kulakov} Если бы кто-нибудь сделал что-то оригинальное, было бы здорово. Я только свиней вырезать умею, да монеты к ним пририсовывать.

\AUTHOR{Artem Trubachev} Я уже рисовал.

\AUTHOR{Vyacheslav Kulakov} Тогда просто макет был - его так никто до ума не довёл.

\AUTHOR{Artem Trubachev} Сделаю себе футболку с таким принтом.

\AUTHOR{Vyacheslav Kulakov} Как сделаешь - в ней приходи. Может пойдёт в массы.

\AUTHOR{Vitaly Bolshakov} Это рыбки?.

\AUTHOR{Artem Trubachev} Да. Январская встреча когда планируется?.

\AUTHOR{Ruslan Siraev} В январе ?.

\AUTHOR{Ivan Grishaev} Да, в конце месяца как обычно.

\AUTHOR{Антон Чикин} Пока точной даты нет.

\AUTHOR{Artem Trubachev} Ивнинг в коворнинг, аутсорсеры.

\AUTHOR{Юра Хрусталев} Удаленка в радость.

\AUTHOR{Artem Trubachev} Юра. Ты летом задвигал про курс молодого бойца.

\AUTHOR{Юра Хрусталев} Дада.

\AUTHOR{Vitaly Bolshakov} Юра не знает про пхп.

\AUTHOR{Юра Хрусталев} Пишу план. Учебный.

\AUTHOR{Artem Trubachev} Когда стартанет?.

\AUTHOR{Юра Хрусталев} Пару месяцев. К весне.

\AUTHOR{Миша Хорпяков} Что за курс?.

\AUTHOR{Юра Хрусталев} Веб дев.

\AUTHOR{Dmitry Sobolev} Что там будет?.

\AUTHOR{Юра Хрусталев} Для менеджеров по продажам.

\AUTHOR{Dmitry Sobolev} Всем привет.

\AUTHOR{Юра Хрусталев} Докер. Питон. Дж. Линукс. Деплой.

\AUTHOR{Sergey Zinoviev} Cap deploy и ниипет :D.

\AUTHOR{Юра Хрусталев} Как теорема?. Cap.

\AUTHOR{Artem Trubachev} Кэп. Очевидный деплой.

\AUTHOR{Юра Хрусталев} Кстати господа.

\AUTHOR{Artem Trubachev} Хорошо бы пару слов о CI  тогда вставить.

\AUTHOR{Юра Хрусталев} 24ого или 25ого будет следующий рефакторинг.

\AUTHOR{Artem Trubachev} В курс.

\AUTHOR{Юра Хрусталев} Обязательгно. Весь спектр. Как в девопсру.

\AUTHOR{Artem Trubachev} Думаю надо 25го делать. Татьянин день. Как никак.

\AUTHOR{Ruslan Siraev} Почем Юра курс.

\AUTHOR{Юра Хрусталев} Верно. Еще нет цены.

\AUTHOR{Антон Чикин} http://giant-penis-license.org/.

\AUTHOR{Денис Ковалев} Изhttps://github.com/edankwan/penis.js.

\AUTHOR{Pavel Barakaev} Чат, здоров! У меня есть доклад на тему смм для разработчиков и стартаперов.Основные темы: MVP на базе соцмедиа, что можно получить, а что нет, как работать с подрядчиками и где они наебывают, пара лайфхаков. С пивом должно пойти. Рассказывать?.

\AUTHOR{Юра Хрусталев} !!!!. 25ого?.

\AUTHOR{Pavel Barakaev} Да, вполне.

\AUTHOR{Eldjarn Ingvarsson} Smm, mvp?. Model/view/presenter:).

\AUTHOR{Юра Хрусталев} Кул. Разгонят нас еще тетеревы.

\AUTHOR{Ruslan Siraev} Да.

\AUTHOR{Pavel Barakaev} Куда ж без тетерев.

\AUTHOR{Vitaly Bolshakov} 25-го в 18-00 дебаты Лебедева с Навальным обещают).

\AUTHOR{Artem Trubachev} Ну они к 19 закончат как раз.

\AUTHOR{Миша Хорпяков} Можно там на большом экране посмотреть. Я буду за Лебедева болеть. Я его кстати один раз видел, на выставке в Ганновере.

\AUTHOR{Антон Чикин} И как тебе?.

\AUTHOR{Ruslan Siraev} Я его видел в москве в 2007. Он толстым был тогда.

\AUTHOR{Антон Чикин} Мне кажется будет уныло. Они оба как-то сдулись.

\AUTHOR{Ruslan Siraev} Думаешь сможешь выступить интереснее чем они?.

\AUTHOR{Антон Чикин} Заходи к нам в офис. Тут каждый день такие выступления. Рефакторинг - это просто расширенная версия наших ежедневных спичей.

\AUTHOR{Ruslan Siraev} К вам это куда?.

\AUTHOR{Eldjarn Ingvarsson} Да, где ваш офис и кто вы?:).

\AUTHOR{Антон Чикин} Макс, ты же нас всех знаешь.

\AUTHOR{Миша Хорпяков} Кстати он и тогда толстый был. Хотя я его тоже в 2006 или 2007 видел.

\AUTHOR{Eldjarn Ingvarsson} Вас знаю, а офис твой нет:). Или ты про ДА.

\AUTHOR{Антон Чикин} Сидим, работаем на удаленке, снимаем офис.

\AUTHOR{Eldjarn Ingvarsson} Я вот подумываю бросить этот чертов интернет магазин и кодить на лиспе за еду.

\AUTHOR{Ruslan Siraev} Какой у тебя магазин. ?.

\AUTHOR{Eldjarn Ingvarsson} Окей. Сайт окейдоставка.

\AUTHOR{Антон Чикин} А зачем?.

\AUTHOR{Eldjarn Ingvarsson} Рексофт делает это, и делает ужасно, потому что платформа мягко говоря старая и гадкая. Но это бизнес, они хотят ibm. А я им нужен потому что заменить некем:).

\AUTHOR{Антон Чикин} Ну отлично. Проси денег и свободы. 20\% на лисп.

\AUTHOR{Eldjarn Ingvarsson} Денег дадут. Свободы пока что нет, но я над этим работаю. Денег я раз в квартал прошу, дают:). Кстати хорошая идея. Насчет просить частичного перевода на пресейл любой. С хорошими техоологиями. Может и сработать. Если выгорит - с меня пьянка в дублине:) или не в Дублине, как скажешь.

\AUTHOR{Антон Чикин} У меня одному незаменимому знакомому предлагали даже баб снимать регулярно за счет конторы.

\AUTHOR{Eldjarn Ingvarsson} На самом деле это фигово когда на проекте есть такие люди. Это и проблема менеджмента, и лидов.

\AUTHOR{Антон Чикин} Это хорошо. Нас поэтому пока не повыпизживали и не заменили студентами филфака.

\AUTHOR{Eldjarn Ingvarsson} Это хорошо только если для тебя приемлемо послать нахер всех когда знаешь что проект от тебя зависит.Тогда ты и сам свободен и руки можешь выкручивать. Мне это не нравится. Мне больше по душе когда как наемник - пришел, сделал, руки пожали и дальше пошел. Но я так пока не умею делать. Чтобы хорошо, просто и можно было передать lore  за умеренное время.

\AUTHOR{Антон Чикин} Да ничего там нельзя быстро передать. Это все сказки из книжек.

\AUTHOR{Ruslan Siraev} Вово.

\AUTHOR{Антон Чикин} Там везде. Везде. Везде. Адские кущи и бардак.

\AUTHOR{Eldjarn Ingvarsson} Ну, пары месяцев нам на прошлом проекте хватало. А он сложный.

\AUTHOR{Антон Чикин} И вот про эту штуку знал только джон, который уволился чтобы открыть свою мастерскую по лозоплетению. Все слеплено из говна.

\AUTHOR{Eldjarn Ingvarsson} Понятно, что все не передашь. И да, процент говна есть. Но обычно он меньше половины:).

\AUTHOR{Антон Чикин} Кроме мега опенсорс проектов, которые пилит 1000 человек в течение 20 лет.

\AUTHOR{Eldjarn Ingvarsson} А у нас около 110\%. О. А ято в опенсорс проектах?. Что.

\AUTHOR{Антон Чикин} Ну там просто человекочасов немеряно вбухано. Нет майлстоунов.

\AUTHOR{Eldjarn Ingvarsson} Там кодобаза адская, там разве не еще хуже?.

\AUTHOR{Антон Чикин} У ядра неплохая кодебаза, из того, что я видел. Дженкинс тоже поддерживают в порядке. Это из тех, в которых я лазил. Докер тоже вроде ничего, но я не углублялся и не люблю подходы Го. Но там во многих либо оч мало профитов - на еду кортима, либо нет профитов вообще.

\AUTHOR{Eldjarn Ingvarsson} Адская не в смысле плохая. Большая очень.

\AUTHOR{Антон Чикин} Так что ситуация немного другая. Там настроены процессы, которые не позволяют говнокодить. Но это ценой человекочасов и отсутствия майлстоунов. Все можно хорошо сделать, если делать неспеша и не за свои деньги.

\AUTHOR{Marat Khusnetdinov} Очень продажи и смм заинтересовало.А где будет следующая сессия?.

\AUTHOR{Денис Ковалев} Голосование еще открыто. https://www.poll-maker.com/poll937052x4fFa44CE-39.

\AUTHOR{Миша Хорпяков} /results@PollBot.

\AUTHOR{Ruslan Siraev} Кому работа))). Дмитрий Провоторов·Заплачу 20 000 рублей тому, кто порекомендует к нам в Мануфактура IT Production & Graphic Design back-end разработчика и мы возьмем его на работу. Есть несколько больших и интересных проектов из области управления данными, киберспорта, гиперлокальных мобильных приложений. Можно несколько. Share! Like!Вакансия Back-end разработчик в Воронеже, работа в Мануфактура СофтВакансия Back-end разработчик. Зарплата: от 80000 руб.. Воронеж. Требуемый опыт: 1–3 года.…VORONEZH.HH.RU. Даже денег с него не возьму).

\AUTHOR{Vitaly Bolshakov} Чет мало платит за рекомендацию).

\AUTHOR{Ruslan Siraev} Обычно платят столько сколько зарплата.

\AUTHOR{Vitaly Bolshakov} Вооо. Вот и я о том же.

\AUTHOR{Ruslan Siraev} Это же воронежская мануфактура. Откуда у него деньги).

\AUTHOR{Pavel Antonov} Это публичная оферта же?.

\AUTHOR{Юра Хрусталев} Говноед блять. Извините. Пусть жопы болнкам за 20к нюхает.

\AUTHOR{Sergey Kharchenko 🦁} Я думаю, за рекомендацию он все-таки платит меньше, чем зп.

\AUTHOR{Vitaly Bolshakov} В смысле, думаешь? Там же написано все).

\AUTHOR{Ruslan Siraev} Ну в фб разместил - значит публичная.

\AUTHOR{Vitaly Bolshakov} 20 против 80. Против от 80).

\AUTHOR{Ruslan Siraev} Кароче пацаны, он же платит за рекомендации а не за факт трудуустройства. Давайте все сходите ).

\AUTHOR{Vitaly Bolshakov} 😆.

\AUTHOR{Ruslan Siraev} И будет отлично бабок получить. На весь год снимем себе пивбар для выступлений.

\AUTHOR{Dmitry Sobolev} А в России есть публичная оферта вообще?.

\AUTHOR{Ruslan Siraev} Да.

\AUTHOR{Dmitry Sobolev} И никак не надо обозначать?.

\AUTHOR{Ruslan Siraev} В магазине обязаны продать по той цене, по которой ценник. Например.

\AUTHOR{Dmitry Sobolev} Типа что написал то и оферта?.

\AUTHOR{Pavel Barakaev} С Димой это не сработает.

\AUTHOR{Ruslan Siraev} Он верит в Бога.

\AUTHOR{Pavel Barakaev}

\AUTHOR{Ruslan Siraev}

\AUTHOR{Юра Хрусталев} Как говорил один мой латвийский коллега "attention whore" этот дима.

\AUTHOR{Миша Хорпяков} Там написано "и мы его возьмём". Так что DDos не прокатит. Ну вообще он перевоспитывается потихонечку, начал з/п писать.

\AUTHOR{Vitaly Bolshakov} )).

\AUTHOR{Dmitry Sobolev} Я всегда просто так людей рекомендовал. Как бы чё, жалко что ли если человек на хорошую работу попадёт.

\AUTHOR{Антон Власов} Если бы он еще начал их платить :)).

\AUTHOR{Ruslan Siraev} Скандалы интриги расследования.

\AUTHOR{Антон Власов} Да ладно, это я так :).

\AUTHOR{Стас М} Давайте уже инсайдов. Можем анонимный чатик для инсайдов сделать.

\AUTHOR{Антон Власов} Какие инсайды, все все знают :).

\AUTHOR{Стас М} Я например нет.

\AUTHOR{Антон Власов} Провоторов еще та балаболка это не секрет. Подпишись на фб :).

\AUTHOR{Стас М} Провоторова?.

\AUTHOR{Антон Власов} Угу. И подобавляй в друзья тех кто у него коментит :).

\AUTHOR{Ruslan Siraev} Было смешно как то наблюдатьза как в егопосте его обвинили в том , что он работу не сделал и бабки не отдал. Ему пишут - он удаляет. Пишут - удаляет.

\AUTHOR{Антон Власов} Ахаха. Там класс история. Но слишком инсайдно :).

\AUTHOR{Ruslan Siraev} Ну взял чувак бабки - не расчитал силы. Бывает.

\AUTHOR{Антон Власов} Ага.

\AUTHOR{Vitaly Bolshakov} 😆.

\AUTHOR{Антон Власов}  Но без пивасика инсайды не пойдут.

\AUTHOR{Юра Хрусталев} Пару скринов из ТКС.

\AUTHOR{Миша Хорпяков} Хорош!.

\AUTHOR{Ruslan Siraev} А ты инвестируешь через ТКС ?.

\AUTHOR{Юра Хрусталев} Да какие там инвестиции. Яж на фрилансе. Еле еле на кинзмараули по пятницам хватает.

\AUTHOR{Антон Чикин} https://journal.tinkoff.ru/gop-stop/?utm_source=vk_tj&utm_medium=dsp.fix&utm_campaign=smm.public.

\AUTHOR{Юра Хрусталев} Когда небезопастно рядом с эколофтом?.

\AUTHOR{Антон Чикин} Убегайте при первой возможностиОтдайте деньги и телефонНе сопротивляйтесь, если вас бьют. Что думает ТКС о своей клиентской базе.

\AUTHOR{Юра Хрусталев} А межтем модульбанк то поднял тарифы. Кроме центрального. И все равно безопастнее , чем альфа на 95\%.

\AUTHOR{Антон Чикин} Твое очко после обращения в тиньков. http://protinkoff.ru/wp-content/uploads/2016/11/tinkoff-loves-you-1.jpg. http://www.vokrug.tv/pic/person/b/e/b/a/medium_beba967527b78b75a515004d29697e17.jpeg.

\AUTHOR{Юра Хрусталев} Мы поняли твою позицию. Хейтер.

\AUTHOR{Антон Чикин} Я разве сказал что-то плохое?.

\AUTHOR{Юра Хрусталев} Меж темhttp://blog.conan.io/2016/09/27/Why-a-C++-package-manager-can\%27t-be-written-in-C++.html?mkt_tok=eyJpIjoiTmpSbU9HRXpaR1V4WVRaayIsInQiOiJUQjRuTURpaFFZVlZ5emtLQlB5aFZpRXpId1B1QUxoU0dydWQzWUpxaTkzS21OZ1wvRlwvbkFRRTVXbUd3MGQ3TUZSWFpCQlwvNGFSWlgwcDBOZjkrWGRHUzhJb1pkQ05hUXBwRWhBbExuMmNcL3lJNTJObGtybzFyZ3haTzRpQmdGejkifQ\%3D\%3D.

\AUTHOR{Антон Чикин} http://www.known.ru/upload/resize_cache/iblock/ffe/500_428_055e56cd252b6bbc6df70d6d68c84b704/ffed405aa39b18836f97c37284817080.jpg.

\AUTHOR{Стас М} Тиньков дядька без комплексов.

\AUTHOR{Антон Чикин} Типичная аудитория. http://sngcredits.ru/uploads/posts/2016-04/1460612752_kredit-nakichnimi-tinkoff-bank-usloviya.png.

\AUTHOR{Юра Хрусталев} Антон, перейди уже к печатному варажению мысли, не через эмодзи с высокой степенью сжатия.

\AUTHOR{Антон Чикин} Все предыдущие бизнесы тинькова после продажи оказались не прибыльными.

\AUTHOR{Юра Хрусталев} Очень интересно.

\AUTHOR{Ruslan Siraev} Погоди и насколько модуль поднял тарифы?.

\AUTHOR{Юра Хрусталев} Письма не приходило?. Максимальный 4.5к теперь. На минимальном комиссия за перевод появилась 0.75\%.

\AUTHOR{Ruslan Siraev} Кхмм нет. Спасибо за инфу.

\AUTHOR{Миша Хорпяков} https://www.youtube.com/watch?v=Czxcqqq7WPs.

\AUTHOR{Vitaly Bolshakov} Но зачем?.

\AUTHOR{Миша Хорпяков} Зачем обзор или клавиатура?.

\AUTHOR{Dmitry Sobolev} Да, где прямое управление из мозга? к чему все эти кнопки?.

\AUTHOR{Vitaly Bolshakov} Обзор то ладно, клавиатуры зачем такие?).

\AUTHOR{Миша Хорпяков} Приходи попробуй.

\AUTHOR{Антон Чикин} Язь здоровенный!.

\AUTHOR{Vitaly Bolshakov} Едрить.

\AUTHOR{Антон Чикин} Поймал.

\AUTHOR{Vitaly Bolshakov} Приду, попробую).

\AUTHOR{Юра Хрусталев} Нужно полную версию подготовить, цензура выризала все политические высказывания.

\AUTHOR{Dmitry Sobolev} Ну это такие как лет 15 назад? Щёлкающие?.

\AUTHOR{Юра Хрусталев} Они.

\AUTHOR{Ivan Grishaev} https://www.youtube.com/watch?v=GBRl2_rcyws.

\AUTHOR{Dmitry Sobolev} Олдскул - круть.

\AUTHOR{Ruslan Siraev} Где ты ее применять WASD будешь?.

\AUTHOR{Dmitry Sobolev} Ну тактильно приятно и звук. Это как мотоцикл с правильным глушителем и мотором )).

\AUTHOR{Ruslan Siraev} Ты про харлей?.

\AUTHOR{Dmitry Sobolev} Ну например. Кому что нарвится.

\AUTHOR{Ruslan Siraev} Для любителей планета ИЖ.

\AUTHOR{Dmitry Sobolev} Мне эндурки нравятся лично.

\AUTHOR{Миша Хорпяков} В слепую печать удобнее. Тактильность классная. В слепую я имею в виду 10-пальцевый способ. + кастомизация. Если надоедят цвета то я могу поменять клавиши.

\AUTHOR{Vitaly Bolshakov} Закажешь новый комплект или как?.

\AUTHOR{Миша Хорпяков} Например.

\AUTHOR{Dmitry Sobolev} Перекрасит.

\AUTHOR{Юра Хрусталев} Вдруг стана сменит флаг. А ты фан флага страны не кейборде.

\AUTHOR{Ruslan Siraev} Как могут надоесть цвета) если ты печатаешь вслепую.

\AUTHOR{Антон Чикин} Мне кажется модераторы скоро забанят Руслана за такие вопросы.

\AUTHOR{Юра Хрусталев} Когда продаешь шеры в стартапах смотришь в клаву, чтобы правильно заполнить лейблы на бирже.

\AUTHOR{Dmitry Sobolev} Руслан, ты женат?.

\AUTHOR{Юра Хрусталев} Пфф. Парень есть?.

\AUTHOR{Dmitry Sobolev} Я скорее к тому что если женат должен знать как цвета могут надоесть.

\AUTHOR{Ruslan Siraev} Дима, увы второе место в моем феррари уже занято. Оно двухместное и для третьего там места нет.

\AUTHOR{Dmitry Sobolev} Жаль-жаль.

\AUTHOR{Ruslan Siraev} Дима ты с какого района будешь?.

\AUTHOR{Dmitry Sobolev} Так феррари то уже занято всё равно.

\AUTHOR{Юра Хрусталев} Ну приора то в гараже стоит.

\AUTHOR{Ruslan Siraev} Приора для заднеприводных пассажиров - мы не такие).

\AUTHOR{Dmitry Sobolev} )) Чё, там, как в Воронеже, сугробы сошли, кстати?.

\AUTHOR{Юра Хрусталев} Сошли, если ты погоду решил обсудить.

\AUTHOR{Dmitry Sobolev} Ну снег скорее.

\AUTHOR{Юра Хрусталев} Но рефакторинг все равно неплохое место. Для обсуждения заднеприводных приор. ;).

\AUTHOR{Dmitry Sobolev} По доккеру вопрос. Может быть  локальный кэш доккера (/var/lib/docker) расшарем между несколькими машинами? Ну например через EFS амазона.

\AUTHOR{Юра Хрусталев} Может. Можешь даже экспортировать.

\AUTHOR{Dmitry Sobolev} Ну чтоб каждая из машин не тянула отдельно доккер имэжд из репозитория. И чё, не будет конфликтов?.

\AUTHOR{Юра Хрусталев} Только тебе нужно монтировать чуть ниже уровнем. Где слои лежат.

\AUTHOR{Dmitry Sobolev} Это как?. Не понял.

\AUTHOR{Mikhail Vyukov}

\AUTHOR{Юра Хрусталев} Dmitry/var/lib/docker/image/devicemapper/. \0.

\AUTHOR{Mikhail Vyukov}

\AUTHOR{Юра Хрусталев}

\AUTHOR{Ivan Grishaev} Почему подписи у Сталина не с грузинским акцентом?.

\AUTHOR{Mikhail Vyukov}  Юрец оценитhttps://cs540100.vk.me/c543101/v543101536/18292/E0hZESA9S48.jpg.

\AUTHOR{Artem Trubachev} Так было же уже. Куча таких новостей. Типа вегатарианка забралась на эверест доказать что веганы все могут, и умерла от недостатка веществ из овощей. http://www.vesti.ru/doc.html?id=2756765&cid=520.

\AUTHOR{Dmitry Sobolev} Ну надо сказать что и невеганы там мрут очень даже нередко.

\AUTHOR{Юра Хрусталев} Я практикующий вегетарианец теперь.

\AUTHOR{Миша Хорпяков} Феминист.

\AUTHOR{Mikhail Vyukov}

\AUTHOR{Dmitry Sobolev} Практикущий это иногда ешь овощи?.

\AUTHOR{Artem Trubachev} Я вчера вечером съел огурец.

\AUTHOR{Юра Хрусталев} Не ем говядину, свинину и курицу.

\AUTHOR{Dmitry Sobolev} Крутой.

\AUTHOR{Artem Trubachev} Я теперь вегетарианец?.

\AUTHOR{Юра Хрусталев} Вряд ли.

\AUTHOR{Artem Trubachev} Я говядину уже неделю не ел. Только индейку.

\AUTHOR{Dmitry Sobolev} Даже вкусноприготовленную не ешь?. Стейк там хороший. Другое мясо?.

\AUTHOR{Юра Хрусталев} Пока что никакой стейк.

\AUTHOR{Dmitry Sobolev} Крутой. Я недавно ел стейк из альпаки и морскую свинку. Я точно не вегетарианец. Хотя фрукты тоже люблю очень.

\AUTHOR{Миша Хорпяков} Цитата: "И, наконец, еще одно направление — производства софтвера и цифрового контента.«Уже есть экспортируемые мультфильмы? Это отлично. Но должны быть и сложные решения, например, программирование коллекторов. Только тогда эта сферу можно будет назвать развитой», — подытожил эксперт.". Оригиналhttp://facto.ru/glavnaya_lenta_novostej/2017/01/ekspert_po_strategicheskomu_planirovaniyu_sprognoziroval_kakie_sfery_voronezhskogo_biznesa_budut_raz/. Это так сказать об уровне местных СМИ.

\AUTHOR{Dmitry Sobolev} Ну к экспортируемым мультфилмам лет 20 шли. На играх натренировались арту.

\AUTHOR{Eldjarn Ingvarsson} "программирование коллекторов" я один нихуха не понял?.

\AUTHOR{Dmitry Sobolev} Лет ещё через 10-15 глядишь и коллекторы будем программить. Раз востребовано в мире.

\AUTHOR{Eldjarn Ingvarsson} Зачем программировать чуваков с битами которые приходят отнимать деньги должников?.

\AUTHOR{Dmitry Sobolev} Автоматизация же!.

\AUTHOR{Dmitry Sobolev} https://www.technologyreview.com/s/603381/ai-software-learns-to-make-ai-software/. У всех есть уже план чем заниматься вместо программирования?.

\AUTHOR{Антон Чикин} Юра, какой рекорд у тебя?. Сколько часов не говорил что ты веган?.

\AUTHOR{Стас М} Я буду тренером в кочалке.

\AUTHOR{Marat Khusnetdinov} Продажи оборонке, или вернусь работать инженером ))). Кабель прокладывать и обжимать?.

\AUTHOR{Ruslan Siraev} Пойду мелочь стрелять к ГЧ.

\AUTHOR{Антон Чикин} Буду парашютным инструктором или в Калифорнию уеду и буду там бомжом.

\AUTHOR{Юра Хрусталев} Стану настоящим веганом.

\AUTHOR{Денис Ковалев} http://cs8.pikabu.ru/post_img/2017/01/20/1/1484863669171264145.jpg. http://cs9.pikabu.ru/post_img/2017/01/20/1/148486366819822946.jpg.

\AUTHOR{Юра Хрусталев} Лапшин довольно интересный комментатор.

\AUTHOR{Стас М} Сноб он.

\AUTHOR{Eldjarn Ingvarsson} Человек пишет на С++, после чего я даю ему кусок кода на Erlang и говорю "поправь".Он отвечаетتذهب اللعنة ويموت هناك!.

\AUTHOR{Ruslan Siraev} )).

\AUTHOR{Mikhail Aksenov} Fuck you go and die there!.

\AUTHOR{Ruslan Siraev} Не профильный вопрос коллеги, может у кого знакомые занимаются мебелью? надо вырезать столешницу для стола под нужный мне размер.

\AUTHOR{Dmitry Sobolev} Из чего вырезать то?.

\AUTHOR{Ruslan Siraev} Лдсп.

\AUTHOR{Dmitry Sobolev} Ну обычно где продают там и режут. В кастораме или там леруа мерлене.

\AUTHOR{Ruslan Siraev} В леруа нет такого размера, я бы уже заказал(. И в кастораме. Тоже нет.

\AUTHOR{Dmitry Sobolev} Тогда не знаю. Дерево я бы помог отпилить наверное. А лдсп - не. Идея для бизнеса - воркшоп с инструментами. Столярный.

\AUTHOR{Ruslan Siraev} Идея для бизнеса, это нормальная инфа обо всех кто сидит на промзоне, что бы гуглить легко.

\AUTHOR{Dmitry Sobolev} Ну отпилить это неинтересно для тех кто мебель делает. Это копейки.

\AUTHOR{Ruslan Siraev} ).

\AUTHOR{Dmitry Sobolev} Еслиб тебе с нуля столешницу бы делали.

\AUTHOR{Миша Хорпяков} На хользунова рядом с Депо 2.

\AUTHOR{Ruslan Siraev} Спасибо Михаил.

\AUTHOR{Vyacheslav Kulakov} На проспекте Труда ещё могу место сказать, если интересно. Я там как раз дверки на кухне переделываю.

\AUTHOR{Ruslan Siraev} Подскажи Вячеслав плиз.

\AUTHOR{Стас М} Артурчик.

\AUTHOR{Artur Muradov} Йоу.

\AUTHOR{Стас М} @ykhrustalevА я вот думаю, а что если перед следующим митапом каждому из этой группы в личку написать? Увеличит явку?.

\AUTHOR{Vyacheslav Kulakov} Это в районе Макс-Мастера. Тел.920-432-94-01, Гена. Скажешь от Славика с Торпедо.

\AUTHOR{Ruslan Siraev} Смотря что напишешь Стас). Слав, спасибо!.

\AUTHOR{Юра Хрусталев} @stasik_mexxпопробуем. 24ого точно встреча. Пока не ясно где.

\AUTHOR{Миша Хорпяков} PHPUnit кто знает? Если тест закрашился, то tearDown() не вызывается же?. Как чистить в таком случае?.

\AUTHOR{Vitaly Bolshakov} Чистить что?. Закрашилось - это фэйлед или упало нахрен?.

\AUTHOR{Антон Чикин} А в PHP есть разница?. Фейлд и эксепшен во время теста по-разному себя ведут?.

\AUTHOR{Artem Trubachev} Чистить видимо базу.

\AUTHOR{Vitaly Bolshakov} Да.

\AUTHOR{Artem Trubachev} Или типа того.

\AUTHOR{Vitaly Bolshakov} Фэйл - это фейл теста. А фатал  - это все поломалось).

\AUTHOR{Миша Хорпяков} Когда runtime exception. Т.е. когда tearDown не отработал.

\AUTHOR{Vitaly Bolshakov} Чисть на сетапе).

\AUTHOR{Artem Trubachev} Так как бы и надо.

\AUTHOR{Vitaly Bolshakov} Можно попробовать костыльно через register_shutdown_function, но это костыльно.

\AUTHOR{Миша Хорпяков} В итоге короче забил. Как вариант было обернуть в транзакцию.

\AUTHOR{Ruslan Siraev} В колхозном баре собрались крестьяне. Тут в бар заходит молодой парень, замечает своих знакомых и - давай с ними бухать по чёрному! Вдруг смотрит: несколько ребят тащат одного набухавшегося друга в соседнюю комнату. Он и спрашивает:- А куда его несут?- Да его трахнут сейчас. Нет женщин у нас, понимаешь ли! Да ты ладно, пей - не обращай внимания...Запил парень дальше... Отрубился. Открывает глаза - глядь, толпа колхозников тащит его куда-то. Как заорет он благим матом:- Нет! Нет! Не надо меня в другую комнату нести!!!А те ему, спокойно так:- Да успокойся ты. Мы тебя уже обратно несём...

\AUTHOR{Юра Хрусталев}

\AUTHOR{Mikhail Vyukov} http://downtown.ru/voronezh/city/8860.

\AUTHOR{Антон Власов} Организаторы лол.

\AUTHOR{Ruslan Siraev} Же сейчас организаторы TEDx в Воронеже ищут спикеров, спонсоров и партнеров. Шаверма на лизюкова. Мебель черноземья.

\AUTHOR{Стас М} Иван Дегтярь. Пытается поднять воронеж с колен.

\AUTHOR{Ruslan Siraev} Молодчик он.

\AUTHOR{Антон Чикин} Ну если у тебя все совсем сhttps://pp.vk.me/c639621/v639621395/21ea/FEisedadr40.jpg.

\AUTHOR{Ruslan Siraev} ).

\AUTHOR{Антон Чикин} Теперь это будет моей любимой фразой.

\AUTHOR{Ruslan Siraev} TEDx пройдет в Воронеже и сейчас активно ищет спикеров. Если у вас есть идеи, достойные распространения — пишите. Давайте на базе Теда проведем дипфакторинг ?). Решим на один раз проблему с помещением.

\AUTHOR{Антон Чикин} Можно продать им спикеров.

\AUTHOR{Ruslan Siraev} Например твой последний доклад ).

\AUTHOR{Антон Чикин} В связи с постом в ЖЖ Лебедева, посвященным дебатам с Навальным и озаглавленном "Пошумим блять!". Вот тут можно ознакомиться с тем, что начинается после этой фразы. https://www.youtube.com/watch?v=MVouMFKmaeI&feature=youtu.be&t=164. И чего я жду от дебатов.

\AUTHOR{Ruslan Siraev} Окси вообще крут в этом батле.

\AUTHOR{Юра Хрусталев} Топил за хованского в тот батл.

\AUTHOR{Антон Чикин} Я в шоке. Как тут опрос создать.

\AUTHOR{Юра Хрусталев} Давай не будем сейчас, все таки скоро понедельник.

\AUTHOR{Антон Чикин} Хочу знать кто еще считает что Оксимирона в этом видео крутым парнем.

\AUTHOR{Ruslan Siraev} Антон ты считаешь того мелкого крутым? поди ты еще и за хиллари бы голосовал, если бы мог?.

\AUTHOR{Антон Чикин} Я считаю, что у нас в школьной раздевалке в 7 классе были баттлы получше.

\AUTHOR{Юра Хрусталев}

\AUTHOR{Денис Ковалев} @achikin,https://www.poll-maker.com- для опросов.

\AUTHOR{Антон Чикин} А потом еще и пиздили по результатам. Поэтому надо было смекалку проявлять.

\AUTHOR{Ruslan Siraev} И быстрые ноги.

\AUTHOR{Антон Чикин} А не хуями собеседника с ходу обкладывать.

\AUTHOR{Ruslan Siraev} Дед, я не понимаю древнепидорский.

\AUTHOR{Антон Чикин} Вроде Оксимирон тоже в приличной школе учился, должен уметь.

\AUTHOR{Юра Хрусталев}

\AUTHOR{Антон Чикин} Хотя говорят он там нихуя не учился. С такими мы даже не батлили.

\AUTHOR{Ruslan Siraev} У него оксфорд за его хилой спиной.

\AUTHOR{Антон Чикин} Мой ровестник кстати. А такие слова плохие говорит.

\AUTHOR{Ruslan Siraev} У него папа профессор. Еврей-профессор. Знал какие книги давать ребенку.

\AUTHOR{Антон Чикин} А мама библиотекарь. Но нихера не помогло.

\AUTHOR{Юра Хрусталев} В чем?.

\AUTHOR{Антон Чикин} В развитии. Вы меня не поймите неправильно.

\AUTHOR{Юра Хрусталев} Всмысле?.

\AUTHOR{Антон Чикин} Я послушал Горгород и оценил.

\AUTHOR{Ruslan Siraev} Если он не черный, то он полюбому делает нормальный реп.

\AUTHOR{Антон Чикин} Пошел посмотреть баттлы.

\AUTHOR{Юра Хрусталев} У него просмотров пока больше , чем у тебя.

\AUTHOR{Антон Чикин} Ожидал там чего-то более сильного. Какого-то ренессанса серебрянного века.

\AUTHOR{Ruslan Siraev} Ты послушай его первые песни и последние. Это небо и земля.

\AUTHOR{Антон Чикин} Подумал вот наконец кто-то понял, что можно читать рэп с хорошими стихами. Вот оно. Поперло.

\AUTHOR{Ruslan Siraev} И вообще АРМИЯ 9КЛАССНИКОВ ошибатся не могут.

\AUTHOR{Антон Чикин} А там хуйпизда. И с таким удовольствием человек эти слова произносит.

\AUTHOR{Ruslan Siraev} Ну Антон, скинешь ссылку когда твои просмотры наберут 1 000.

\AUTHOR{Антон Чикин} Как будто вчера узнал. Этого никогда не будет.

\AUTHOR{Ruslan Siraev} И телки трусы будут кидать.

\AUTHOR{Антон Чикин} Разве что на гитхабе.

\AUTHOR{Ruslan Siraev} Во время твоих выступлений.

\AUTHOR{Антон Чикин} Я не хочу чтобы такие телки кидали в меня трусы.

\AUTHOR{Юра Хрусталев} На гитхабе много лайков не набрать.

\AUTHOR{Антон Чикин} Это вот ваши мечты? 10000 лайков и женские трусы на голове?.

\AUTHOR{Ruslan Siraev} Не знаю чем вы там кидались в школьной раздевалке. У нас хотя бы есть мечты).

\AUTHOR{Антон Чикин} Типа пруф оф саксесс?.

\AUTHOR{Юра Хрусталев} Пока что ачиверов не наблюдается.

\AUTHOR{Ruslan Siraev} Разговор перешел в русло « а чего добился ты?».

\AUTHOR{Юра Хрусталев} Я не хотел. Уже понедельник кстати.

\AUTHOR{Антон Чикин} Ну не знаю.

\AUTHOR{Ruslan Siraev} Антон уже пнедельник, давай спать. с утра на работу.

\AUTHOR{Юра Хрусталев} Оксюморон конечно заслуживает уважения, но сообщество милениалов не может вынести его успеха молча.

\AUTHOR{Антон Чикин} После учёбы в Оксфорде Мирон переехал жить в Ист-Энд и начал поиски работы.

\AUTHOR{Sergey Kharchenko 🦁} /stat@combot.

\AUTHOR{Combot} Combot.org/chat/-1001074278604.

\AUTHOR{Антон Чикин} Еще раз. После учёбы в Оксфорде Мирон переехал жить в Ист-Энд и начал поиски работы. И еще раз. После учёбы в Оксфорде Мирон переехал жить в Ист-Энд и начал поиски работы.

\AUTHOR{Ruslan Siraev} Че не так то?).

\AUTHOR{Антон Чикин} И еще раз. После учёбы в Оксфорде Мирон переехал жить в Ист-Энд и начал поиски работы.

\AUTHOR{Ruslan Siraev} Ну объясни мне я вот не понимаю.

\AUTHOR{Юра Хрусталев} Еще один повтор и я пожалуюсь модератору.

\AUTHOR{Антон Чикин} Деньги откуда?. Добился тоже. Без работы.

\AUTHOR{Ruslan Siraev} Так ты прочитал куда он поступил?на самый просто факультет. Чет там древняя литература.

\AUTHOR{Vitaly Bolshakov} Едрить все сложно то.

\AUTHOR{Антон Чикин} В Оксфорд. Оксфорд известное место.

\end{document}
