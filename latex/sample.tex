
\documentclass[10pt]{book}

\usepackage[T2A]{fontenc}
\usepackage{geometry}
\usepackage[utf8]{inputenc}
\usepackage[russian]{babel}

\geometry{
  marginparwidth=12mm,
  papersize={145mm,205mm}, % 145x205 no bleed
  top    = 15mm,           %
  right  = 17mm,           %
  bottom = 25mm,           %
  left   = 23mm            %
}

\usepackage[hyperindex=false]{hyperref}

\setlength{\parskip}{0.32em}

\newcommand{\AUTHOR}[1]{\emph{#1}:}

\newenvironment{teaser}{\vspace{-3em}\slshape}{\vspace{1em}}

\begin{document}

\chapter{Исходники Кложи}

\begin{teaser}
В которой выясняется, что исходники Кложи оформлены не по гайду: скобки
расставлены не так, как принято в крупных фирмах. Скобки и отступы сказываются
на работе кода. Сорваны и другие покровы.
\end{teaser}

\AUTHOR{Антон Чикин} https://darklang.com/ К разговору про серверлесс.

\AUTHOR{Андрей Иванов} Слушал доклад про кложакли? )

\AUTHOR{Антон Чикин} Нет.

\AUTHOR{Андрей Иванов} Там просто тоже этот даркланг упоминали.

\AUTHOR{Антон Чикин} Ну так в чате писали потом. Что там было интересного?

\AUTHOR{Андрей Иванов} А я не знаю, я сам не слушал ) Потом в записи посмотрю
наверное.

\AUTHOR{Антон Чикин} Ну я думаю такое там....

\AUTHOR{Андрей Саксонов} если кто iPhone SE собирался покупать, могу сделать 34-35к за
64гб в мвидео, остальные версии надо смотреть цену, но схема та же. Ну прогони
любой большой проект. hazelcast, mariadb-client, spring. Как там с код
стайлом. Плюс кодстайл иногда меняется по объективным причинам. Стали лучше
инструменты, появились новые фичи в языке. Но старые файлы никто насильно не
переформатирует. Если файл потрогал с другими изменениями, то тогда да.

\AUTHOR{Антон Чикин} Ну это очевидно любому более-менее вменяемому человеку,
который видел сорцы хотя бы пары долгоиграющих проектов. У нас был проект 10+
лет где отступ в старых сорцах был три пробела и трогать их было строго-настрого
запрещено.

\AUTHOR{Дима} Странно, что в джаве за 25 лет, единообразия удалось
достичь. Видимо невменяемые. Ну или посто не привыкли говно в репозитории
держать.

\AUTHOR{Миша Хорпяков} Код Рича в репозитории~--- Рухнама кложуристов. Никто не
смеет осквернять его светлый и идеальный код, несущий благодать словно лучи
утренней зари.

\AUTHOR{Андрей Саксонов} И над сколькими 25 летними кодовыми базами ты работал.

\AUTHOR{Михаил Козачков} А кто привык?

\AUTHOR{Дима} Ноль, а ты? https://t.me/deeprefactoring/391252

\AUTHOR{Андрей Саксонов} Парочку припоминаю.

\AUTHOR{Дима} 25 лет? Расскажи пожалуйста. Очень интересно.

\AUTHOR{Андрей Саксонов} Ну там Java 1.1. Ну там монолитик под IBM System Z. Не, ну
она в продакшен в районе 1999 пошла, до этого там чет обновляли.

\AUTHOR{Антон Чикин} Когда ты лезешь в старую кодебазу, пытаешься разобраться,
делаешь там blame и видишь там <<Джуниор девелопер Сидоров, фиксед кодестайл>>~---
это нихрена не здорово. Хочешь сказать в компании которая настолько стабильна и
успешна что сумела просуществовать 20 лет на одной кодебазе? Думаю это
охуенно. На самом деле в любой другой индустрии иди расскажи что ты не изучал
проекты 10-20 летней давности~--- тебя на работу не возьмут никуда.

\AUTHOR{Илья Чурсин} Обратная сторона медали, когда ты лезешь такой по blame
назад, а там и полезная работа сделана, и код-стайл сменен (ну потому что влезли
же). Хрен разберешься тогда что и где. Уж лучше еще раз сделать blame от
отдельного коммита со стайлом, чем разгребать кучу отступов от полезного.

\AUTHOR{Андрей Саксонов} Да это тоже надо аккуратно делать. Если код стайл настолько
поменялся что твой автоформат распидорасит весь файл, то лучше не надо.

\AUTHOR{Ivan Grishaev} Код стайл надо менять очень осторожно и сто раз подумать.

\end{document}
